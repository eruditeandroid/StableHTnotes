\documentclass{article}

\usepackage{notes}

\title{Stable Homotopy Theory}

\begin{document}

\maketitle

\tableofcontents

\pagebreak

\section{1/30 (Joe Hlavinka) -- Organizational Meeting and Introduction}

\subsection{Logistics}

The seminar page can be found on \href{https://math.berkeley.edu/~kskapoor/seminars/spring2025/stableht}{Kabir Kapoor's website}.

Here are some topics we'd like to cover:

\begin{enumerate}
	\item Basics of stable homotopy theory (the Hurewicz theorem for spectra, Freudenthal suspension)
	\item The Pontryagin-Thom theorem (possibly following Milnor's treatment in \emph{Topology from the Differentiable Viewpoint})
	\item Segal's interpretation of spectra using configuration spaces and electromagnetism
	\item Infinite loop spaces and the equivalence between connective spectra and grouplike $\EE_\infty$-spaces
	\item Thomason's equivalence between $(-1)$-connective spectra and (a localization of) symmetric monoidal categories
	\item Complex bordism $\MU$, formal group laws, and Quillen's geometric interpretation
	\item Brown-Peterson spectra and ``designer'' Thom-ish spectra
	\item Chromatic homotopy theory (Morava $K$-theories, etc)
	\item The Landweber exact functor theorem
	\item The Adams spectral sequence as descent along $\Spec \ZZ \to \Spec \SS$
	\item Stable cohomology operations (Steenrod squares, etc)
	\item Floer homotopy theory
	\item Symplectic / spectral Khovanov homology
	\item Rational homotopy theory
	\item Integral Kirwan surjectivity
	\item Algebraic $K$-theory (after Nikolaus-Scholze?)
	\item Geometric representation theory with coefficients in spectra
	\item Spectral algebraic theory
\end{enumerate}

The goal is to be leisurely and friendly.

\subsection{Introduction -- Higher Algebra}

We like cohomology theories (as axiomatized e.g.\ by Eilenberg and Steenrod).
In particular, we enjoy thinking about stable cohomology operations, multiplicative structures, and many other things we can do on cohomology.
If we try to think of cohomology theories as some sort of functors from the category $\Sc$ of spaces to abelian groups, we end up with a bad ``category of cohomology theories'' -- the result isn't triangulated.
Instead, it's better to work with the category $\Sp$ of \emph{spectra}.

\begin{thm}
	The category $\Sp$ is the ``triangularization'' of the category of cohomology theories.
	In particular, the objects of $\Sp$ are cohomology theories, but there are more maps of spectra, called \emph{phantom maps},\footnote{In particular, there is a phantom map $\Hrm\QQ \to \KU$.
	There are no phantom maps between ``Landweber exact'' spectra (in particular, those that appear in chromatic homotopy theory).} which disappear on taking homotopy classes.
\end{thm}

Given a cohomology theory $F$, the \emph{Brown representability theorem} allows us to find a spectrum $F_X$ such that, for any space $Y$, we have
\[
	F(Y) \simeq \Hom(\Sigma^\infty Y, F_X).
\]

Another viewpoint on spectra is that $\Sp = \Sc_*[\Sigma\inv]$, i.e.\ spectra are what we get when we invert the suspension functor of pointed spaces.
This is closely related to the Spanier-Whitehead construction of spectra.
In general, this relationship is much less subtle than the relationship between spectra and cohomology theories.
Thinking of spectra as cohomology theories (with the above caveats), the above claim comes from the suspension axiom for cohomology theories: for a spectrum $X$ and a space $Y$, we have
\[
	\Hom_\Sp(\Sigma^\infty Y, X) \simeq X^*(Y),
\]

\subsection{Stable homotopy}

For a pointed space $Y$, there are natural maps $\pi_n(Y) \to \pi_{n+1}(\Sigma Y)$ (ultimately coming from the adjunction between suspension and loop spaces).\footnote{There are some subtleties -- any subcategory of the classical category of topological spaces with such an adjunction has very few colimits.
This can be avoided with enough model- or $\infty$-categorical dexterity.}

\begin{thm}[Freudenthal suspension]
	For $i \gg 0$, the maps $\pi_{n+i}(\Sigma^i Y) \to \pi_{n+i+1}(\Sigma^{i+1} Y)$ are equivalences.
\end{thm}

Thus the sequence of maps eventually stabilizes, and we may define:

\begin{dfn}
	For $n \in \ZZ$, let the $n$th \emph{stable homotopy group} of $Y$ be $\pi_n^s(Y) = \colim_i \pi_{n+i}(\Sigma^i Y)$.
	From the equivalence $\pi_n^s(Y) \simeq \pi_{n+1}^s(\Sigma Y)$, we see that $\pi_*^s(-)$ is a homology theory.
\end{dfn}

For a general spectrum $X$, we can write $X_*(Y) \simeq \pi_*^s(X \wedge \Sigma^\infty Y)$.
This gives a notion of \emph{homology theory} associated to any spectrum

The functor $\pi_*^s$ is corepresented by a spectrum $\SS$, called the \emph{sphere spectrum}.
That is,
\[
	\pi_*^s(Y) = \Hom_*(\SS, \Sigma^\infty Y),
\]
and $\pi_n^s(Y)$ consists of homotopy classes of maps $\SS[n] \to \Sigma^\infty Y$.
We can write $\SS = \Sigma^\infty S^0$.

\subsection{Stable $\infty$-categories}

We should think of the category of spectra as a \emph{stable $\infty$-category}.
In a 1-category, we would have a pushout diagram
\[
	\begin{tikzcd}
		X \rar \dar & \pt \dar \\
		\pt \rar & \pt.
	\end{tikzcd}
\]

Things are different in $\infty$-categories: we are allowed to replace objects by equivalent objects.
Thus, replacing $\pt$ by the (contractible) cone $CX$, we obtain a pushout diagram
\[
	\begin{tikzcd}
		X \rar \dar & \pt \dar \\
		\pt \rar & \Sigma X.
	\end{tikzcd}
\]
Similarly, using the path space fibration, we get a pullback diagrams
\[
	\begin{tikzcd}
		\Omega X \rar \dar & \pt \dar \\
		\pt \rar & X.
	\end{tikzcd}
\]
In $\Sp$, we have $\Omega = \Sigma\inv$.
Stability is a generalization of this: we want pushout squares to agree with pullback squares.

\section{2/6 (Ward Veltman) -- The Pontryagin-Thom Construction}

\subsection{Motivation: Steenrod realization}

The \emph{Steenrod realization problem} asks: if $X$ is a topological space and $\alpha \in H_n(X; \FF_2)$, is there a compact $n$-manifold $M$ and $f: M \to X$ such that $f_* [M] = \alpha$?
The answer turns out to be yes if we work with $\FF_2$ coefficients but no if we work with $\ZZ$ coefficients.

Note that, if $M = \partial W$ and $f: M \to X$ extends to $F: W \to X$, then
\[
	f_* [M] = f_* \partial [W] = 0
\]
because $f_*$ factors through $H_{n+1}(W, M) \to H_{n+1}(X, X) = 0$.
Thus cobordism should be relevant to this story somehow -- we should try to understand it more.

\subsection{Cobordism and statement of Thom's theorem}

\begin{dfn}
	Let $M$ and $N$ be smooth closed $n$-manifolds.
	We say that $M$ and $N$ are \emph{cobordant} if there exists a compact $(n+1)$-manifold $W$ such that $M \sqcup N = \partial W$.
	This forms an equivalence relation, and we denote the set of equivalence classes by $\Nc_n$.
	Then $\Nc = \oplus_{n \in \NN} \Nc_n$ forms a ring (where addition is $\sqcup$ and multiplication is $\times$).
\end{dfn}

We can also consider cobordism with different stable structures on the tangent bundles: these give rings $\Nc_n^G$, where $G$ is the relevant structure group.

\begin{thm}[Thom, 1954]
	There is a ring isomorphism 
	\[
		\Nc_* \xrightarrow{\sim} \pi_*(\MO) \cong \FF_2[x_k : k \neq 2^\ell - 1].
	\]
\end{thm}

Our goal for today will be to construct this map.
We should first explain what $\pi_*(\MO)$ is.

Let $\BO(k) = \Gr_k(\RR^\infty)$ be the classifying space of $\Orm(k)$.
Write $\EO(k) = V_k(\RR^\infty)$ for the universal principal $\Orm(k)$-bundle and $\gamma_k^\infty = \EO(k) \times^{\Orm(k)} \RR^k$ for the universal vector bundle (both defined over $\BO(k)$).
Let
\[
	\MO(k) = \Th(\gamma_k^\infty) := D(\gamma_k^\infty) / S(\gamma_k^\infty),
\]
where $D(-)$ is the disk bundle, $S(-)$ is the sphere bundle, and $\Th(-)$ is the \emph{Thom space}.
The inclusions $i: \BO(k) \hookrightarrow \BO(k+1)$ (induced by $\Gr_k(\RR^{n+k}) \hookrightarrow \Gr_{k+1}(\RR^{n+k+1})$) fit into commutative squares
\[
	\begin{tikzcd}
		\gamma_k^\infty \oplus \RR \rar \dar & \gamma_{k+1}^\infty \dar \\
		\BO(k) \rar & \BO(k+1).
	\end{tikzcd}
\]
On Thom spaces, we have
\[
	\Th(\gamma_k^\infty \oplus \RR) \simeq \Th(\gamma_k^\infty) \wedge S^1 \simeq \Sigma \MO(k).
\]
Thus we have natural maps $\sigma_k: \Sigma \MO(k) \to \MO(k+1)$, and the spaces $\MO(k)$ assemble into a spectrum (or ``prespectrum'' according to one's choice of model).

\begin{dfn}
	The \emph{$n$th stable homotopy group of $\MO$} is defined to be
	\[
		\pi_n(\MO) = \colim_k \pi_{n+k}(\MO(k)).
	\]
\end{dfn}

\subsection{Construction of the map}

We want to define a map $\Nc_n \to \pi_n(\MO)$.
Let $M^n$ represent a class in $\Nc_n$.
Choose an embedding $i: M \hookrightarrow \RR^{n+k}$ (for some $k$) such that $i$ extends to an embedding of a tubular neighborhood $N$ around $i(M)$.
We may assume that $N = D_\epsilon(\nu_M)$ is a disk bundle (of some radius $\epsilon > 0$).
Note that $\ol{N} / \partial \ol{N} \cong \Th(\nu_M)$.

The \emph{Pontryagin-Thom collapse map} $\theta: S^{n+k} \to \Th(\nu_M)$ is defined as follows.
View $S^{n+k} \cong \RR{n+k} \cup \{\infty\}$.
Then $\theta$ maps $N$ identically into $\ol{N} / \partial \ol{N}$ and maps $S^{n+k} \setminus N$ to $\partial \ol{N} / \partial \ol{N} = \pt$.
This map is continuous.

Consider the \emph{Gauss map} $M \to \Gr_k(\RR^{n+k})$ given by $x \mapsto (T_x M)^\perp$.
The composite $M \to \Gr_k(\RR^{n+k}) \to \BO(k)$ is (by construction) the classifying map for the normal bundle $\nu_M$.
Considering the map $\nu_M \to \gamma_k^\infty$ and passing to Thom spaces, we obtain a map $\eta: \Th(\nu_M) \to \MO(k)$

Composing these maps gives $\eta \circ \theta: S^{n+k} \in \MO(k)$, hence a class $[\eta \circ \theta] \in \pi_n(\MO)$.
Note that this depends on various choices:
\begin{itemize}
	\item A representative $M^n$
	\item An embedding $i: M \to \RR^{n+k}$
	\item A tubular neighborhood $N$ of $i(M)$
\end{itemize}
We'll need to show that this is independent of all of the choices.

To see independence of $M$, it suffices to show that if $M = \partial W^{n+1}$, then the corresponding class in $\pi_n(\MO)$ is zero.
For this, embed $S^{n+k} \hookrightarrow \RR^{n+k+1}$ as the unit sphere, and extend $i$ to a map $i': N' \to \ol{D}_1(\RR^{k+1})$ for some tubular neighborhood $N'$ of $W$.
(We can do this if $k$ is large enough because $M$ and $W$ are compact.)
We end up with a commutative diagram 
\[
	\begin{tikzcd}
		S^{n+k} \ar[rr] \dar["\theta_M"] & & \ol{D}_1(\RR^{n+k+1}) \dar["\theta_W"] \\
		\Th(\nu_M) \ar[rr] \ar[dr, "\eta_M"] & & \Th(\nu_W) \ar[dl, "\eta_W"] \\
		 & \MO(k) &
	\end{tikzcd}
\]
Since $\eta_M \circ \theta_M$ factors through $\ol{D}_1(\RR^{n+k+1}) \simeq \pt$, we have $[\eta_M \circ \theta_M] = 0$.

We'll skip showing the independence of the other choices.

\subsection{The map is a group homomorphism}

We'd like to show that the map $\Phi: \Nc_n \to \pi_n(\MO)$ is a group homomorphism.

We first recall the group structure on $\pi_n(\MO)$.
Given classes $[\phi], [\psi] \in \pi_n(\MO)$, pick representatives $\phi, \psi \in \pi_{n+k}(\MO(k))$.
The sum $[\phi] + [\psi]$ is defined by the map
\[
	S^{n+k} \xrightarrow{c} S^{n+k} \vee S^{n+k} \xrightarrow{\phi \vee \psi} \MO(k).
\]

To see that $\Phi$ is a group homomorphism, we need to show that $\Phi([M_1 \sqcup M_2]) = \Phi([M_1]) + \Phi([M_2])$.
Pick embeddings $i_j: M_j \to \RR^{n+k}$ (for $j = 1, 2$) such that the $N_j$ lie in distinct hemispheres of $S^{n+k}$.
Write $\nu_1 = \nu_{M_1}$, $\nu_2 = \nu_{M_2}$ and $\nu_{12} = \nu_{M_1 \sqcup M_2}$.
Then $\Th(\nu_{12}) \cong \Th(\nu_1) \wedge \Th(\nu_2)$, and $\theta_{12}$ restricts to $\theta_1$ on $M_1$ and $\theta_2$ on $M_2$.
Using this and the definitions, we see the desired claim.

\subsection{The inverse map}

To get an inverse map $\pi_*(\MO) \to \Nc_*$, let $\psi: S^{n+k} \to \MO(k)$ represent a class in $\pi_n(\MO)$.
Then the image of $\psi$ lies in $\Th(\gamma_k^{k + \ell}$ for some finite $\ell$.
Consider the image of the zero section $\sigma: \Gr_k(\RR^{k+\ell}) \to \Th(\gamma_k^{k + \ell})$.

By transversality, we may homotope $\psi$ to a smooth map $\tilde{\psi}$ transverse to $\sigma(\Gr_k(\RR^{k+\ell}))$.
Then $M = \tilde{\psi}\inv(\sigma(\Gr_k(\RR^{k+\ell}))$ is a smooth $n$-manifold, i.e.\ an element of $\Nc_*$.
As above, one needs to show that this construction is independent of the choices made.
It also takes a bit of work to show that this is a homotopy inverse to the above map.

\subsection{(Gabriel Beiner) -- Postscript on stable homotopy theory}

When working fully with spectra, we can view a version of Pontryagin-Thom as saying that the sphere spectrum is a Thom spectrum.
More precisely, this relies on the notion of a stable normal framing.

\begin{dfn}
	A \emph{stable normal framing} of a manifold $M$ is an embedding $i: M \to \RR^{n+k}$ such that $\nu_M \cong \RR^k \times M$, considered up to isotopy and stabilization.
\end{dfn}

If we let $\Omega_n^\st(\pt)$ be the group of cobordism classes of stably normal framed $n$-manifolds, then Pontryagin-Thom tells us that $\Omega_n^\st(\pt) \cong \pi_n^\st(\pt)$.
At the level of spectra, $\SS$ is the Thom spectrum of stably normal framed $n$-manifolds.

One can use this to compute homotopy groups of spheres!
For example, $\pi_1^\st = \ZZ / 2 \ZZ$ corresponds to the two stable normal framings of $S^1$.
One can also see that $\pi_3^\st = \ZZ / 24 \ZZ$ is generated by a K3 surface (noting that it has Euler characteristic $\chi = 24$, giving an upper bound on the order of the group, and using a separate argument to show that the order is actually 24).

\section{2/13 (Jacob Erlikhman) -- Pontryagin-Thom and Configuration Spaces}

Last week, Ward discussed the Pontryagin-Thom isomorphism.
Given an embedding $M^k \hookrightarrow \RR^{n+k}$, the cobordism class of $M^k$ gives an element of $\pi_k \MO$.
More generally, using the Pontryagin-Thom collapse map, we see that, for any $X^{n+k}$, the \emph{cohomotopy classes} in $\pi^{n+k}(X)$ correspond to \emph{framed cobordism classes} in $X$.
This week, we will interpret $\pi^n(X)$ in terms of configuration spaces of points and the Pontryagin-Thom collapse map as an ``electric field map.''

\subsection{Configuration spaces and electric fields}

\begin{dfn}
	For a topological space $X$ and $k \in \NN$, let $\Conf_k(X) = (X^k \setminus \Delta^k_X) / S_k$.
	We write $C(X) = \sqcup_k \Conf_k(X)$.
	For $X = \RR^n$, we write $C_n = C(\RR^n)$.
\end{dfn}

In physics, given a particle of charge $q$ at $r' \in \RR^n$, we obtain an electric field $E'_{r'}: \RR^n \setminus \{r'\} \to \RR^n$ given by
\[
	E'_{r'}(r) = \frac{q}{d(r, r')^3} (r - r').
\]
This extends to $E'_{r'}: S^n \to S^n$ if we let $E'_{r'}(r') = \infty$ and $E'_{r'}(\infty) = 0$.
We can homotope this map so that $r \mapsto \infty$ if $d(r, r') \geq \epsilon$ (for some fixed $\epsilon$).
In particular, this new map (call it $E_r$) maps the basepoint $\infty$ to $\infty$

Given a collection $c$ of particles at points $r'_1, \dots, r'_k \in \RR^n$, we get an electric field $E_c: S^n \to S^n$.
This defines a map
\begin{align*}
	C_n &\to \Omega^n S^n \\
	c &\mapsto E_c.
\end{align*}

We may replace $C_n$ by a monoid $C_n'$ defined as follows.
For $s, t \in \RR$, let $\RR^n_{s,t} = (\RR \setminus [s, t]) \times \RR^{n-1}$.
Let $T_t: \RR^n_{0,t'} \to \RR^n_{t,t'+t}$ be the translation map in the first factor.
As sets, let $C_n' = \bset{(c, t) \in C_n \times \RR}{t \geq 0, c \in \RR^n_{0,t}}$.
The monoid operation is $(c, t) \cdot (c', t') = (c \cup T_t c', t + t')$.

We want to prove the following.

\begin{thm}
	$B C_n' \simeq \Omega^{n-1} S^n$.
\end{thm}

\begin{cor}
	$C_n \simeq \Omega^n S^n$.
\end{cor}

\subsection{Pontryagin-Thom collapse}

We defined the Pontryagin-Thom collapse map as follows.
Given $i: M^k \hookrightarrow \RR^{n+k}$ with tubular neighborhood $N$, we view $\RR^{n+k} \subset S^{n+k}$ and define $\Pi_M: S^{n+k} \to \Th(\nu_M) = S^{n+k} / (S^{n+k} \setminus N)$.

\begin{ex}
	Suppose $M = c \in C_n$, viewed as a subset of $\RR^n$.
	Then $\Pi_M: S^n \to \vee_\ell S^n$ (sending everything outside disks around points of $c$ to the basepoint of $\vee_\ell S^n$).
	The map $E_c: S^n \to S^n$ is homotopic to the composite of $\Pi_M$ with the coproduct map $\vee_\ell S^n \to S^n$.
	This is easiest to see by drawing pictures.
\end{ex}

\subsection{Generalization to other $X$}

To extend these claims to other spaces $X$, we need a technical condition.

\begin{dfn}
	Say that $X$ has a \emph{good basepoint} $0 \in X$ if there exists a homotopy $h_t: X \to X$ such that $h_0 = \id_X$, $h_t(0) = 0$, and $h_1\inv(0)$ is a neighborhood of $0$.\footnote{Essentially: we want a neighborhood of $0$ to be contractible to $0$ via a \emph{global} homotopy.}
\end{dfn}

This is automatically satisfied for any manifold.
Throughout we will assume $X$ has a good basepoint $0$.

\begin{dfn}
	Let $C_n(X)$ be the \emph{configuration space of points in $\RR^n$ labeled by $X$}, i.e.\
	\[
		C_n(X) = \{ (c \in C_n, x: c \to X) \} / \sim
	\]
	where $(c, x) \sim (c', x')$ if $c \subset c'$ and
	\[
		x'(r) = \begin{cases}
			x(r) & r \in c \\
			0 & r \not\in c.
		\end{cases}
	\]
\end{dfn}

\begin{thm}
	If $X$ is path-connected, then $C_n(X) \simeq \Omega^n \Sigma^n X$.
\end{thm}

Taking $X = S^0$, we obtain the corollary of the previous theorem.

\begin{proof}[Partial proof of theorem]
	One defines a monoid version $C'_n(X)$ and proves $BC_n'(X) \simeq C_{n-1}(\Sigma X)$.
	(This is hard.)
	Then one obtains
	\begin{align*}
		C_{n-1}(\Sigma X) &\simeq BC'_n(X) \\
				  &\simeq \Omega BC'_{n-1}(\Sigma X) \\
				  &\simeq \Omega C_{n-2}(\Sigma^2 X) \\
				  &\simeq \dots \\
				  &\simeq \Omega^{n-1} C_0(\Sigma^n X) \simeq \Omega^{n-1}(\Sigma^n X).
	\end{align*}
\end{proof}

\begin{rmk}
	Joe explained that this can be interpreted in terms of $\infty$-category theory: if $F_k$ is the ``free $\EE_k$-algebra'' functor, then
	\[
		F_{n-1}(X) \simeq \onebb \otimes_{F_{n}(\Sigma X)} \onebb.
	\]
\end{rmk}

Given a closed framed $M^k \hookrightarrow X^{n+k}$, we can define a Pontryagin-Thom collapse map $\Pi: X \to \Th(\nu_M)$.
This also gives an ``electric field'' map producing framed cohomotopy classes.

\subsection{Annihilating particles}

Let $M$ be a smooth manifold, and fix a submanifold $M_0 \subset M$.
Write $C(M, M_0; X)$ for the configuration space of points in $M$ labeled by $X$ which are allowed to annihilate points that live in $M_0$:
\[
	(m_1, \dots, m_k, x_1, \dots, x_k) \sim (m_1, \dots, m_{k-1}, x_1, \dots, x_{k-1}) \textrm{ if } m_k \in M_0.
\]

Let $W$ be an $m$-manifold without boundary containing $M$.
Write $\hat{T} W$ for the fiberwise compactification of $TW$, and write $\hat{\tau}$ for the fiberwise smash product of $\hat{T} W$ with $X$.
Thus the fibers of $\hat{\tau} \to W$ are $\hat{T}_w W \wedge X \simeq \Sigma^m X$.
We have a section $s_\infty: W \to \hat{\tau}$.

\begin{dfn}
	Given $N_0 \subset N \subset W$, let $\Gamma(N, N_0; X)$ be the space of sections $s$ of $\hat{\tau}$ such that $s|_{N_0} = s_\infty$.
\end{dfn}

\begin{thm}
	If $M$ is compact, then $C(M, M_0; X) \xrightarrow{\sim} \Gamma(W \setminus M_0, W \setminus M; X)$.
\end{thm}

\end{document}
