\documentclass{article}

\usepackage{notes}

\title{Stable Homotopy Theory}

\begin{document}

\maketitle

\tableofcontents

\pagebreak

\section{1/30 (Joe Hlavinka) -- Organizational Meeting and Introduction}

\subsection{Logistics}

The seminar page can be found on \href{https://math.berkeley.edu/~kskapoor/seminars/spring2025/stableht}{Kabir Kapoor's website}.

Here are some topics we'd like to cover:

\begin{enumerate}
	\item Basics of stable homotopy theory (the Hurewicz theorem for spectra, Freudenthal suspension)
	\item The Pontryagin-Thom theorem (possibly following Milnor's treatment in \emph{Topology from the Differentiable Viewpoint})
	\item Segal's interpretation of spectra using configuration spaces and electromagnetism
	\item Infinite loop spaces and the equivalence between connective spectra and grouplike $\EE_\infty$-spaces
	\item Thomason's equivalence between $(-1)$-connective spectra and (a localization of) symmetric monoidal categories
	\item Complex bordism $\MU$, formal group laws, and Quillen's geometric interpretation
	\item Brown-Peterson spectra and ``designer'' Thom-ish spectra
	\item Chromatic homotopy theory (Morava $K$-theories, etc)
	\item The Landweber exact functor theorem
	\item The Adams spectral sequence as descent along $\Spec \ZZ \to \Spec \SS$
	\item Stable cohomology operations (Steenrod squares, etc)
	\item Floer homotopy theory
	\item Symplectic / spectral Khovanov homology
	\item Rational homotopy theory
	\item Integral Kirwan surjectivity
	\item Algebraic $K$-theory (after Nikolaus-Scholze?)
	\item Geometric representation theory with coefficients in spectra
	\item Spectral algebraic theory
\end{enumerate}

The goal is to be leisurely and friendly.

\subsection{Introduction -- Higher Algebra}

We like cohomology theories (as axiomatized e.g.\ by Eilenberg and Steenrod).
In particular, we enjoy thinking about stable cohomology operations, multiplicative structures, and many other things we can do on cohomology.
If we try to think of cohomology theories as some sort of functors from the category $\Sc$ of spaces to abelian groups, we end up with a bad ``category of cohomology theories'' -- the result isn't triangulated.
Instead, it's better to work with the category $\Sp$ of \emph{spectra}.

\begin{thm}
	The category $\Sp$ is the ``triangularization'' of the category of cohomology theories.
	In particular, the objects of $\Sp$ are cohomology theories, but there are more maps of spectra, called \emph{phantom maps},\footnote{In particular, there is a phantom map $\Hrm\QQ \to \KU$.
	There are no phantom maps between ``Landweber exact'' spectra (in particular, those that appear in chromatic homotopy theory).} which disappear on taking homotopy classes.
\end{thm}

Given a cohomology theory $F$, the \emph{Brown representability theorem} allows us to find a spectrum $F_X$ such that, for any space $Y$, we have
\[
	F(Y) \simeq \Hom(\Sigma^\infty Y, F_X).
\]

Another viewpoint on spectra is that $\Sp = \Sc_*[\Sigma\inv]$, i.e.\ spectra are what we get when we invert the suspension functor of pointed spaces.
This is closely related to the Spanier-Whitehead construction of spectra.
In general, this relationship is much less subtle than the relationship between spectra and cohomology theories.
Thinking of spectra as cohomology theories (with the above caveats), the above claim comes from the suspension axiom for cohomology theories: for a spectrum $X$ and a space $Y$, we have
\[
	\Hom_\Sp(\Sigma^\infty Y, X) \simeq X^*(Y),
\]

\subsection{Stable homotopy}

For a pointed space $Y$, there are natural maps $\pi_n(Y) \to \pi_{n+1}(\Sigma Y)$ (ultimately coming from the adjunction between suspension and loop spaces).\footnote{There are some subtleties -- any subcategory of the classical category of topological spaces with such an adjunction has very few colimits.
This can be avoided with enough model- or $\infty$-categorical dexterity.}

\begin{thm}[Freudenthal suspension]
	For $i \gg 0$, the maps $\pi_{n+i}(\Sigma^i Y) \to \pi_{n+i+1}(\Sigma^{i+1} Y)$ are equivalences.
\end{thm}

Thus the sequence of maps eventually stabilizes, and we may define:

\begin{dfn}
	For $n \in \ZZ$, let the $n$th \emph{stable homotopy group} of $Y$ be $\pi_n^s(Y) = \colim_i \pi_{n+i}(\Sigma^i Y)$.
	From the equivalence $\pi_n^s(Y) \simeq \pi_{n+1}^s(\Sigma Y)$, we see that $\pi_*^s(-)$ is a homology theory.
\end{dfn}

For a general spectrum $X$, we can write $X_*(Y) \simeq \pi_*^s(X \wedge \Sigma^\infty Y)$.
This gives a notion of \emph{homology theory} associated to any spectrum

The functor $\pi_*^s$ is corepresented by a spectrum $\SS$, called the \emph{sphere spectrum}.
That is,
\[
	\pi_*^s(Y) = \Hom_*(\SS, \Sigma^\infty Y),
\]
and $\pi_n^s(Y)$ consists of homotopy classes of maps $\SS[n] \to \Sigma^\infty Y$.
We can write $\SS = \Sigma^\infty S^0$.

\subsection{Stable $\infty$-categories}

We should think of the category of spectra as a \emph{stable $\infty$-category}.
In a 1-category, we would have a pushout diagram
\[
	\begin{tikzcd}
		X \rar \dar & \pt \dar \\
		\pt \rar & \pt.
	\end{tikzcd}
\]

Things are different in $\infty$-categories: we are allowed to replace objects by equivalent objects.
Thus, replacing $\pt$ by the (contractible) cone $CX$, we obtain a pushout diagram
\[
	\begin{tikzcd}
		X \rar \dar & \pt \dar \\
		\pt \rar & \Sigma X.
	\end{tikzcd}
\]
Similarly, using the path space fibration, we get a pullback diagrams
\[
	\begin{tikzcd}
		\Omega X \rar \dar & \pt \dar \\
		\pt \rar & X.
	\end{tikzcd}
\]
In $\Sp$, we have $\Omega = \Sigma\inv$.
Stability is a generalization of this: we want pushout squares to agree with pullback squares.

\section{2/6 (Ward Veltman) -- The Pontryagin-Thom Construction}

\subsection{Motivation: Steenrod realization}

The \emph{Steenrod realization problem} asks: if $X$ is a topological space and $\alpha \in H_n(X; \FF_2)$, is there a compact $n$-manifold $M$ and $f: M \to X$ such that $f_* [M] = \alpha$?
The answer turns out to be yes if we work with $\FF_2$ coefficients but no if we work with $\ZZ$ coefficients.

Note that, if $M = \partial W$ and $f: M \to X$ extends to $F: W \to X$, then
\[
	f_* [M] = f_* \partial [W] = 0
\]
because $f_*$ factors through $H_{n+1}(W, M) \to H_{n+1}(X, X) = 0$.
Thus cobordism should be relevant to this story somehow -- we should try to understand it more.

\subsection{Cobordism and statement of Thom's theorem}

\begin{dfn}
	Let $M$ and $N$ be smooth closed $n$-manifolds.
	We say that $M$ and $N$ are \emph{cobordant} if there exists a compact $(n+1)$-manifold $W$ such that $M \sqcup N = \partial W$.
	This forms an equivalence relation, and we denote the set of equivalence classes by $\Nc_n$.
	Then $\Nc = \oplus_{n \in \NN} \Nc_n$ forms a ring (where addition is $\sqcup$ and multiplication is $\times$).
\end{dfn}

We can also consider cobordism with different stable structures on the tangent bundles: these give rings $\Nc_n^G$, where $G$ is the relevant structure group.

\begin{thm}[Thom, 1954]
	There is a ring isomorphism 
	\[
		\Nc_* \xrightarrow{\sim} \pi_*(\MO) \cong \FF_2[x_k : k \neq 2^\ell - 1].
	\]
\end{thm}

Our goal for today will be to construct this map.
We should first explain what $\pi_*(\MO)$ is.

Let $\BO(k) = \Gr_k(\RR^\infty)$ be the classifying space of $\Orm(k)$.
Write $\EO(k) = V_k(\RR^\infty)$ for the universal principal $\Orm(k)$-bundle and $\gamma_k^\infty = \EO(k) \times^{\Orm(k)} \RR^k$ for the universal vector bundle (both defined over $\BO(k)$).
Let
\[
	\MO(k) = \Th(\gamma_k^\infty) := D(\gamma_k^\infty) / S(\gamma_k^\infty),
\]
where $D(-)$ is the disk bundle, $S(-)$ is the sphere bundle, and $\Th(-)$ is the \emph{Thom space}.
The inclusions $i: \BO(k) \hookrightarrow \BO(k+1)$ (induced by $\Gr_k(\RR^{n+k}) \hookrightarrow \Gr_{k+1}(\RR^{n+k+1})$) fit into commutative squares
\[
	\begin{tikzcd}
		\gamma_k^\infty \oplus \RR \rar \dar & \gamma_{k+1}^\infty \dar \\
		\BO(k) \rar & \BO(k+1).
	\end{tikzcd}
\]
On Thom spaces, we have
\[
	\Th(\gamma_k^\infty \oplus \RR) \simeq \Th(\gamma_k^\infty) \wedge S^1 \simeq \Sigma \MO(k).
\]
Thus we have natural maps $\sigma_k: \Sigma \MO(k) \to \MO(k+1)$, and the spaces $\MO(k)$ assemble into a spectrum (or ``prespectrum'' according to one's choice of model).

\begin{dfn}
	The \emph{$n$th stable homotopy group of $\MO$} is defined to be
	\[
		\pi_n(\MO) = \colim_k \pi_{n+k}(\MO(k)).
	\]
\end{dfn}

\subsection{Construction of the map}

We want to define a map $\Nc_n \to \pi_n(\MO)$.
Let $M^n$ represent a class in $\Nc_n$.
Choose an embedding $i: M \hookrightarrow \RR^{n+k}$ (for some $k$) such that $i$ extends to an embedding of a tubular neighborhood $N$ around $i(M)$.
We may assume that $N = D_\epsilon(\nu_M)$ is a disk bundle (of some radius $\epsilon > 0$).
Note that $\ol{N} / \partial \ol{N} \cong \Th(\nu_M)$.

The \emph{Pontryagin-Thom collapse map} $\theta: S^{n+k} \to \Th(\nu_M)$ is defined as follows.
View $S^{n+k} \cong \RR{n+k} \cup \{\infty\}$.
Then $\theta$ maps $N$ identically into $\ol{N} / \partial \ol{N}$ and maps $S^{n+k} \setminus N$ to $\partial \ol{N} / \partial \ol{N} = \pt$.
This map is continuous.

Consider the \emph{Gauss map} $M \to \Gr_k(\RR^{n+k})$ given by $x \mapsto (T_x M)^\perp$.
The composite $M \to \Gr_k(\RR^{n+k}) \to \BO(k)$ is (by construction) the classifying map for the normal bundle $\nu_M$.
Considering the map $\nu_M \to \gamma_k^\infty$ and passing to Thom spaces, we obtain a map $\eta: \Th(\nu_M) \to \MO(k)$

Composing these maps gives $\eta \circ \theta: S^{n+k} \in \MO(k)$, hence a class $[\eta \circ \theta] \in \pi_n(\MO)$.
Note that this depends on various choices:
\begin{itemize}
	\item A representative $M^n$
	\item An embedding $i: M \to \RR^{n+k}$
	\item A tubular neighborhood $N$ of $i(M)$
\end{itemize}
We'll need to show that this is independent of all of the choices.

To see independence of $M$, it suffices to show that if $M = \partial W^{n+1}$, then the corresponding class in $\pi_n(\MO)$ is zero.
For this, embed $S^{n+k} \hookrightarrow \RR^{n+k+1}$ as the unit sphere, and extend $i$ to a map $i': N' \to \ol{D}_1(\RR^{k+1})$ for some tubular neighborhood $N'$ of $W$.
(We can do this if $k$ is large enough because $M$ and $W$ are compact.)
We end up with a commutative diagram 
\[
	\begin{tikzcd}
		S^{n+k} \ar[rr] \dar["\theta_M"] & & \ol{D}_1(\RR^{n+k+1}) \dar["\theta_W"] \\
		\Th(\nu_M) \ar[rr] \ar[dr, "\eta_M"] & & \Th(\nu_W) \ar[dl, "\eta_W"] \\
		 & \MO(k) &
	\end{tikzcd}
\]
Since $\eta_M \circ \theta_M$ factors through $\ol{D}_1(\RR^{n+k+1}) \simeq \pt$, we have $[\eta_M \circ \theta_M] = 0$.

We'll skip showing the independence of the other choices.

\subsection{The map is a group homomorphism}

We'd like to show that the map $\Phi: \Nc_n \to \pi_n(\MO)$ is a group homomorphism.

We first recall the group structure on $\pi_n(\MO)$.
Given classes $[\phi], [\psi] \in \pi_n(\MO)$, pick representatives $\phi, \psi \in \pi_{n+k}(\MO(k))$.
The sum $[\phi] + [\psi]$ is defined by the map
\[
	S^{n+k} \xrightarrow{c} S^{n+k} \vee S^{n+k} \xrightarrow{\phi \vee \psi} \MO(k).
\]

To see that $\Phi$ is a group homomorphism, we need to show that $\Phi([M_1 \sqcup M_2]) = \Phi([M_1]) + \Phi([M_2])$.
Pick embeddings $i_j: M_j \to \RR^{n+k}$ (for $j = 1, 2$) such that the $N_j$ lie in distinct hemispheres of $S^{n+k}$.
Write $\nu_1 = \nu_{M_1}$, $\nu_2 = \nu_{M_2}$ and $\nu_{12} = \nu_{M_1 \sqcup M_2}$.
Then $\Th(\nu_{12}) \cong \Th(\nu_1) \wedge \Th(\nu_2)$, and $\theta_{12}$ restricts to $\theta_1$ on $M_1$ and $\theta_2$ on $M_2$.
Using this and the definitions, we see the desired claim.

\subsection{The inverse map}

To get an inverse map $\pi_*(\MO) \to \Nc_*$, let $\psi: S^{n+k} \to \MO(k)$ represent a class in $\pi_n(\MO)$.
Then the image of $\psi$ lies in $\Th(\gamma_k^{k + \ell})$ for some finite $\ell$.
Consider the image of the zero section $\sigma: \Gr_k(\RR^{k+\ell}) \to \Th(\gamma_k^{k + \ell})$.

By transversality, we may homotope $\psi$ to a smooth map $\tilde{\psi}$ transverse to $\sigma(\Gr_k(\RR^{k+\ell}))$.
Then $M = \tilde{\psi}\inv(\sigma(\Gr_k(\RR^{k+\ell}))$ is a smooth $n$-manifold, i.e.\ an element of $\Nc_*$.
As above, one needs to show that this construction is independent of the choices made.
It also takes a bit of work to show that this is a homotopy inverse to the above map.

\subsection{(Gabriel Beiner) -- Postscript on stable homotopy theory}

When working fully with spectra, we can view a version of Pontryagin-Thom as saying that the sphere spectrum is a Thom spectrum.
More precisely, this relies on the notion of a stable normal framing.

\begin{dfn}
	A \emph{stable normal framing} of a manifold $M$ is an embedding $i: M \to \RR^{n+k}$ such that $\nu_M \cong \RR^k \times M$, considered up to isotopy and stabilization.
\end{dfn}

If we let $\Omega_n^\st(\pt)$ be the group of cobordism classes of stably normal framed $n$-manifolds, then Pontryagin-Thom tells us that $\Omega_n^\st(\pt) \cong \pi_n^\st(\pt)$.
At the level of spectra, $\SS$ is the Thom spectrum of stably normal framed $n$-manifolds.

One can use this to compute homotopy groups of spheres!
For example, $\pi_1^\st = \ZZ / 2 \ZZ$ corresponds to the two stable normal framings of $S^1$.
One can also see that $\pi_3^\st = \ZZ / 24 \ZZ$ is generated by a K3 surface (noting that it has Euler characteristic $\chi = 24$, giving an upper bound on the order of the group, and using a separate argument to show that the order is actually 24).

\section{2/13 (Jacob Erlikhman) -- Pontryagin-Thom and Configuration Spaces}

Last week, Ward discussed the Pontryagin-Thom isomorphism.
Given an embedding $M^k \hookrightarrow \RR^{n+k}$, the cobordism class of $M^k$ gives an element of $\pi_k \MO$.
More generally, using the Pontryagin-Thom collapse map, we see that, for any $X^{n+k}$, the \emph{cohomotopy classes} in $\pi^{n+k}(X)$ correspond to \emph{framed cobordism classes} in $X$.
This week, we will interpret $\pi^n(X)$ in terms of configuration spaces of points and the Pontryagin-Thom collapse map as an ``electric field map.''

\subsection{Configuration spaces and electric fields}

\begin{dfn}
	For a topological space $X$ and $k \in \NN$, let $\Conf_k(X) = (X^k \setminus \Delta^k_X) / S_k$.
	We write $C(X) = \sqcup_k \Conf_k(X)$.
	For $X = \RR^n$, we write $C_n = C(\RR^n)$.
\end{dfn}

In physics, given a particle of charge $q$ at $r' \in \RR^n$, we obtain an electric field $E'_{r'}: \RR^n \setminus \{r'\} \to \RR^n$ given by
\[
	E'_{r'}(r) = \frac{q}{d(r, r')^3} (r - r').
\]
This extends to $E'_{r'}: S^n \to S^n$ if we let $E'_{r'}(r') = \infty$ and $E'_{r'}(\infty) = 0$.
We can homotope this map so that $r \mapsto \infty$ if $d(r, r') \geq \epsilon$ (for some fixed $\epsilon$).
In particular, this new map (call it $E_r$) maps the basepoint $\infty$ to $\infty$

Given a collection $c$ of particles at points $r'_1, \dots, r'_k \in \RR^n$, we get an electric field $E_c: S^n \to S^n$.
This defines a map
\begin{align*}
	C_n &\to \Omega^n S^n \\
	c &\mapsto E_c.
\end{align*}

We may replace $C_n$ by a monoid $C_n'$ defined as follows.
For $s, t \in \RR$, let $\RR^n_{s,t} = (\RR \setminus [s, t]) \times \RR^{n-1}$.
Let $T_t: \RR^n_{0,t'} \to \RR^n_{t,t'+t}$ be the translation map in the first factor.
As sets, let $C_n' = \bset{(c, t) \in C_n \times \RR}{t \geq 0, c \in \RR^n_{0,t}}$.
The monoid operation is $(c, t) \cdot (c', t') = (c \cup T_t c', t + t')$.

We want to prove the following.

\begin{thm}
	$B C_n' \simeq \Omega^{n-1} S^n$.
\end{thm}

\begin{cor}
	$C_n \simeq \Omega^n S^n$.
\end{cor}

\subsection{Pontryagin-Thom collapse}

We defined the Pontryagin-Thom collapse map as follows.
Given $i: M^k \hookrightarrow \RR^{n+k}$ with tubular neighborhood $N$, we view $\RR^{n+k} \subset S^{n+k}$ and define $\Pi_M: S^{n+k} \to \Th(\nu_M) = S^{n+k} / (S^{n+k} \setminus N)$.

\begin{ex}
	Suppose $M = c \in C_n$, viewed as a subset of $\RR^n$.
	Then $\Pi_M: S^n \to \vee_\ell S^n$ (sending everything outside disks around points of $c$ to the basepoint of $\vee_\ell S^n$).
	The map $E_c: S^n \to S^n$ is homotopic to the composite of $\Pi_M$ with the coproduct map $\vee_\ell S^n \to S^n$.
	This is easiest to see by drawing pictures.
\end{ex}

\subsection{Generalization to other $X$}

To extend these claims to other spaces $X$, we need a technical condition.

\begin{dfn}
	Say that $X$ has a \emph{good basepoint} $0 \in X$ if there exists a homotopy $h_t: X \to X$ such that $h_0 = \id_X$, $h_t(0) = 0$, and $h_1\inv(0)$ is a neighborhood of $0$.\footnote{Essentially: we want a neighborhood of $0$ to be contractible to $0$ via a \emph{global} homotopy.}
\end{dfn}

This is automatically satisfied for any manifold.
Throughout we will assume $X$ has a good basepoint $0$.

\begin{dfn}
	Let $C_n(X)$ be the \emph{configuration space of points in $\RR^n$ labeled by $X$}, i.e.\
	\[
		C_n(X) = \{ (c \in C_n, x: c \to X) \} / \sim
	\]
	where $(c, x) \sim (c', x')$ if $c \subset c'$ and
	\[
		x'(r) = \begin{cases}
			x(r) & r \in c \\
			0 & r \not\in c.
		\end{cases}
	\]
\end{dfn}

\begin{thm}
	If $X$ is path-connected, then $C_n(X) \simeq \Omega^n \Sigma^n X$.
\end{thm}

Taking $X = S^0$, we obtain the corollary of the previous theorem.

\begin{proof}[Partial proof of theorem]
	One defines a monoid version $C'_n(X)$ of $C_n(X)$ and proves $BC_n'(X) \simeq C_{n-1}(\Sigma X)$.
	(This is hard.)
	Then one obtains
	\begin{align*}
		C_{n-1}(\Sigma X) &\simeq BC'_n(X) \\
				  &\simeq \Omega BC'_{n-1}(\Sigma X) \\
				  &\simeq \Omega C_{n-2}(\Sigma^2 X) \\
				  &\simeq \dots \\
				  &\simeq \Omega^{n-1} C_0(\Sigma^n X) \simeq \Omega^{n-1}(\Sigma^n X). \qedhere
	\end{align*}
\end{proof}

\begin{rmk}
	Joe explained that this can be interpreted in terms of $\infty$-category theory: if $F_k$ is the ``free $\EE_k$-algebra'' functor, then
	\[
		F_{n-1}(X) \simeq \onebb \otimes_{F_{n}(\Sigma X)} \onebb.
	\]
\end{rmk}

Given a closed framed $M^k \hookrightarrow X^{n+k}$, we can define a Pontryagin-Thom collapse map $\Pi: X \to \Th(\nu_M)$.
This also gives an ``electric field'' map producing framed cohomotopy classes.

\subsection{Annihilating particles}

Let $M$ be a smooth manifold, and fix a submanifold $M_0 \subset M$.
Write $C(M, M_0; X)$ for the configuration space of points in $M$ labeled by $X$ which are allowed to annihilate points that live in $M_0$:
\[
	(m_1, \dots, m_k, x_1, \dots, x_k) \sim (m_1, \dots, m_{k-1}, x_1, \dots, x_{k-1}) \textrm{ if } m_k \in M_0.
\]

Let $W$ be an $m$-manifold without boundary containing $M$.
Write $\hat{T} W$ for the fiberwise compactification of $TW$, and write $\hat{\tau}$ for the fiberwise smash product of $\hat{T} W$ with $X$.
Thus the fibers of $\hat{\tau} \to W$ are $\hat{T}_w W \wedge X \simeq \Sigma^m X$.
We have a section $s_\infty: W \to \hat{\tau}$.

\begin{dfn}
	Given $N_0 \subset N \subset W$, let $\Gamma(N, N_0; X)$ be the space of sections $s$ of $\hat{\tau}$ such that $s|_{N_0} = s_\infty$.
\end{dfn}

\begin{thm}
	If $M$ is compact, then $C(M, M_0; X) \xrightarrow{\sim} \Gamma(W \setminus M_0, W \setminus M; X)$.
\end{thm}

\section{2/20 (Zechen Bian) -- Introduction to Spectra}

I missed this day.
If you have notes and would like to share, please contact me!

\section{2/27 (Joe Hlavinka) -- Brave New Algebra}

\subsection{How to think about spectra}

We should think of spectra as \emph{modules over the sphere spectrum}: in symbols, $\Sp = \Mod(\SS)$.
Recall that, given $E \in \Sp$, we have
\[
	E_i(X) = \pi_i(\ul{\Map}(\SS, E \otimes X)) = \pi_0(\ul{\Map}(\SS[i], E \otimes X))
\]
where $\otimes$ is the smash product of spectra.
Shifts here are defined by $X[1] = X \otimes \Sigma^\infty S^1$.
If we model spectra by $\Omega$-spectra, then we can compute the inverse shift $[-1]$ by $(X_0, X_1, \dots)[-1] = (X_1, X_2, \dots)$.
Homotopy groups of spectra can be computed via
\[
	\pi_i(X) = \pi_0(\Omega^\infty(X[i])).
\]

The category of spectra satisfies Whitehead's theorem: the functor $\pi_*$ is conservative (if $\pi_*(f)$ is an equivalence, then $f$ is an equivalence).
However, $\pi_*$ is not fully faithful: there are maps of spectra which are nontrivial but induce trivial maps on homotopy groups.

\begin{ex}
	$\pi_* \End(H\FF_2)$ is the Steenrod algebra (controlling stable cohomology operations in $\FF_2$-cohomology): most maps here are nontrivial but trivial on homotopy groups.
\end{ex}

\begin{ex}
	Stabilizing the Hopf map gives a non-nullhomotopic map $\eta: \SS[1] \to \SS$.
\end{ex}

The smash product of spectra is inexplicit, but it has a few useful properties:
\begin{itemize}
	\item $\SS \otimes X \simeq X$.
	\item $X \otimes (Y[i]) \simeq (X \otimes Y)[i]$.
	\item The functor $\Sigma^\infty$ is symmetric monoidal: $\Sigma^\infty(X \wedge Y) \simeq \Sigma^\infty X \otimes \Sigma^\infty Y$.
	\item $\otimes$ preserves colimits in each variable.
	\item $\otimes$ has a right adjoint $\ul{\Map}$: $\Map(X \otimes Y, Z) \simeq \Map(X, \ul{\Map}(Y, Z))$.
\end{itemize}

\subsection{Ring spectra}

\begin{dfn}
	A spectrum $X$ is a \emph{ring spectrum} if there exists a multiplication map $\mu: X \otimes X \to X$ together with homotopy coherence data for the usual associativity diagrams.
\end{dfn}

\begin{dfn}
	A ring spectrum $X$ is an \emph{$\EE_n$-ring spectrum} if, for all $m$, the $m$-fold multiplication $X^{\otimes m} \to X$ factors through
	\[
		X^{\otimes m} \to (X^{\otimes m} \otimes \Conf_m(\RR^n))_{hS_m}
	\]
	where $\Conf_m(\RR^n)$ is the space of ordered collections of $m$ points in $\RR^n$, and $(-)_{hS_m}$ is the homotopy quotient by the usual $S_m$-action.\footnote{That is, $(-)_{hS_m}$ is a $BS_m$-indexed colimit in $\Sp$.}
	We also require that these factorizations come with coherence data.
\end{dfn}

\begin{rmk}
	We are being cavalier here: one should really talk about \emph{operads} to make sense of this.
	The data we've given is not enough in general
\end{rmk}

The notion of ``weak $\EE_n$-ring spectrum'' captures some sort of commutativity of the multiplication.
Note that commutativity is \emph{data}: we must encode not just that two elements commute with each other but \emph{how they do so}.
We want to interpolate between complete noncommutativity ($\EE_1$) and a contractible space of ways to be commutative ($\EE_\infty$).

If $X$ is an $\EE_1$-algebra, then $\pi_*(X)$ is an associative ring.
If $X$ is an $\EE_2$-algebra, then $\pi_*(X)$ is furthermore commutative and also comes with interesting extra operations.

\begin{ex}
	$\SS$ is an $\EE_\infty$-ring spectrum.
\end{ex}

\begin{ex}
	$\SS / p = \cofib(\SS \xrightarrow{p} \SS)$ is not even a ring spectrum.
	For example, $\pi_0(\SS / 2) = \ZZ / 2$, so if $\SS / 2$ had a ring structure, every element of $\pi_*(\SS / 2)$ would have to be $2$-torsion.
	However, $\pi_0(\SS / 2) = \ZZ / 4 \ZZ$ (as one can see using Toda brackets), giving a contradiction.
\end{ex}

\begin{ex}
	The spectrum $\FF_2 = H\FF_2$ is $\EE_\infty$.
	The composite $X \to X \otimes X \to \FF_2 \otimes F_2 \to \FF_2$ classifies $\phi \mapsto \phi \cup \phi$ in cohomology.
	Factorizing the two-fold multiplication map ends up introducing a way to act by $H^*(\RR\PP^\infty, \FF_2) = \FF_2[[z]]$, which controls the Steenrod squares $\Sq$.
	For $\deg \phi = n$, the Steenrod square $\Sq_{n-i}(\phi)$ is given by $\mu^*(\phi \cdot z^i)$.
\end{ex}

\section{3/6 (Gabriel Beiner) -- Generalized Cohomology Operations and the Atiyah-Hirzebruch Spectral Sequence}

\subsection{Cohomology Operations and Steenrod Algebras}

Let $E$ be a spectrum with cohomology theory $E^*$ and homology theory $E_*$.

\begin{dfn}
	An \emph{$E$-cohomology operation} is a natural transformation $\phi: E^i \to E^{i+n}$.
	A \emph{stable cohomology operation} is a family of cohomology operations (of the same degree) $\{ \phi_i \}_{i \in \ZZ}$ satisfying $\Sigma \circ \phi_i = \phi_{i+1} \circ \Sigma$.
\end{dfn}

A stable cohomology operation is the same as an endomorphism $\phi: E \to E$ in $\Sp$.
By the Yoneda lemma, the algebra of stable cohomology operations is $E^*(E)$.

\begin{dfn}
	The \emph{$E$-Steenrod algebra} is $E_*(E)$.
	One often prefers to consider the dual $E_*(E) = \pi_*(E \wedge E)$, as this is better behaved in general.
\end{dfn}

\begin{thm}
	If $E$ is a nice ring spectrum, then $(\pi_* E, E_* E)$ is a \emph{Hopf algebroid}.
\end{thm}

\begin{ex}
	If $E = HR$ is an Eilenberg-Mac Lane spectrum, then we are considering classical cohomology operations.
	Over $\QQ$, there are no interesting cohomology operations because $H\QQ$ is the rationalization of $\SS$.
	Thus, over $\ZZ$, all of the cohomology operations come from Bockstein maps and torsion operations.
	(The Bockstein map is $\beta: H(\ZZ/p) \to H\ZZ[1]$ induced from $H\ZZ \xrightarrow{p} H\ZZ$.)
	
	To compute the Steenrod algebra of $H\ZZ/p$, we seek to compute $H^*(K(\ZZ/p, *), \ZZ/p)$.
	This can be computed using the Serre spectral sequence for the path space fibration
	\[
		\begin{tikzcd}
			K(\ZZ/p, n - 1) \rar & \pt \rar & K(\ZZ/p, n)
		\end{tikzcd}
	\]

	For $p = 2$, performing this calculation shows that the $H\ZZ/2$-Steenrod algebra $\Ac^*$ is generated by the \emph{Steenrod squares} $\Sq^i: H^n(X; \ZZ/2) \to H^{n+i}(X; \ZZ/2)$, which are characterized by
	\begin{enumerate}
		\item $\Sq^0 = \Id$.
		\item For $\deg(x) < i$, $\Sq^i(x) = 0$.
		\item For $\deg(x) = i$, $\Sq^i(x) = x^2$.
		\item (Cartan formula) $\Sq^n(x \cup y) = \sum_{i=0}^n \Sq^i(x) \cup \Sq^{n-i}(y)$.
	\end{enumerate}
	These satisfy the \emph{Adem relations}, which can be written out explicitly.
	One sees that the dual $\Ac_*$ is a commutative Hopf algebra.
	It is also possible to show that $\Sq^1 = \beta_2$, the mod 2 Bockstein.

	Over $\ZZ/p$, we can show that the $H\ZZ/p$-Steenrod algebra $\Ac_p^*$ is generated by the \emph{$p$th power operations} alongside the mod $p$ Bockstein.
\end{ex}

\begin{ex}
	For $E = \KU$, there are \emph{Adams operations} $\phi^k$ satisfying $\psi^k(\Lc) = \Lc^{\otimes k}$ for line bundles $\Lc$.
	This is extended to arbitrary classes using the splitting principle.
	One can show that these satisfy
	\begin{enumerate}
		\item $\psi^k(x + y) = \psi^k(x) + \psi^k(y)$.
		\item $\psi^k(xy) = \psi^k(x) \psi^k(y)$.
		\item $\psi^k \circ \psi^\ell = \psi^{k \ell}$.
	\end{enumerate}
	However, these operations are \emph{unstable} (except $\psi^{\pm 1}$).

	By work of Adams and Clarke, it turns out that $\KU_*(\KU)$ is quite complicated.
	In particular, $\KU_*(\KU)$ is a free module over $\pi_*(\KU)$ with infinitely many generators, so $\KU^*(\KU)$ is uncountable.
	The complexity of $\KU^*(\KU)$ can be predicted using chromatic homotopy theory.
\end{ex}

\begin{ex}
	By work of Landweber and Novikov, $\MU_*(\MU) = \pi_*(\MU)[b_1, b_2, \dots]$ where $\deg(b_i) = 2i$.
	It follows that $\MU^*(\MU)$ has a ``formal basis'' consisting of Landweber-Novikov operations $S_\omega$ indexed by integer partitions $\omega$.
	These satisfy $S_n \circ S_n = (n + 1) S_{2n} + 2 S_{(n,n)}$.
\end{ex}

\begin{ex}
	After $p$-localizing, $\MU$ becomes a wedge of copies of the Brown-Peterson spectrum $\BP$ (which also relates to Morava $K$-theories).
	We get $\BP_*(\BP) = \pi_*(\BP)[t_1, t_2, \dots]$ where $\deg(t_i) = 2(p^i - 1)$.
	Here $\pi_*(\BP) = \ZZ_{(p)}[v_1, \dots,]$.
	The Steenrod algebra $\BP^*(\BP)$ has a formal basis consisting of Quillen operations $r_\omega$.
\end{ex}

\subsection{Postnikov towers and $k$-invariants}

We'll spoil the punchline now: the $k$-invariants in the Postnikov tower of a spectrum are the differentials in the Atiyah-Hirzebruch spectral sequence.
Let's unpack what this means.

Given a CW complex $X$ on which $\pi_1$ acts trivially on all higher homology groups,\footnote{This assumption can be weakened if we work with local systems} we can form a space $X_n$ by killing off the $i$th homotopy groups for $i > n$ (by attaching cells).
We have maps $X_n \to X_{n-1}$, which we can take to be fibrations.
The homotopy fibers of these maps are Eilenberg-Mac Lane spaces $K(\pi_n, n)$.

We can use this to inductively build $X$ by taking fibrations over each $X_i$.
We start with $X_1$, and then consider $X_2 \to X_1$ as a fibration with fiber $K(\pi_2, 2)$.
This is classified by a map $k^1: X \to BK(\pi_2, 2) = K(\pi_2, 3)$, called a \emph{$k$-invariant}.
Note that $k^1$ may be viewed as an element of $H^3(X_1; \pi_2)$.
More generally, $X^{i+1} \to X_i$ is classified by the \emph{Postnikov $k$-invariant} $k^i \in H^{i+2}(X_i, \pi_{i+1})$.
The $k$-invariants are the fundamental obstructions to trivializing $X_{i+1} \to X_i$.

\begin{thm}
	In this setting, we have $X \simeq \varprojlim X_i$.
	In particular, $X$ is determined by its homotopy groups and Postnikov $k$-invariants.
\end{thm}

Note that $k^1$ (for a fixed space $X$) is a cohomology operation (since it gives a map $K(\pi_1, 1) \to K(\pi_2, 3)$).
The same is not quite true for $k^2$: the map $K(\pi_1, 1) \to K(\pi_3, 4)$ is only well-defined as a map $\ker k_1 \to \coker(K(\pi_2, 2) \to K(\pi_3, 4)$.
We can say $k^2$ is a \emph{secondary cohomology operation}: it is well-defined as a map from the kernel of one cohomology operation to the cokernel of another.
Similarly, $k^3$ is a ``tertiary cohomology operation,'' and more generally $k^i$ gives a partially defined cohomology operation.

This gives a general strategy for constructing cohomology operations.
In this way we can construct Massey products, Toda brackets, and residues of Adem relations.

\begin{ex}
	For a high dimensional sphere $S^n$, the EM spaces appearing in the Postnikov tower are $K(\ZZ, n)$, $K(\ZZ/2, n+1)$, $K(\ZZ/2, n+2), K(\ZZ/24, n+4), \dots$.
	Letting $\iota_n \in H^n(K(\ZZ/p, n); \ZZ/p)$, we have:
	\begin{enumerate}
		\item $k^1 = \Sq^2 \iota_n \in H^{n+2}(K(\ZZ, n); \ZZ/2)$
		\item $k^2 = \Sq^2 \iota_{n+1} \in H^{n+3}(K(\ZZ/2, n+1); \ZZ/2)$
		\item $k^3 = \Sq^4 \iota_n + P_3^1 \iota_n \in H^{n+4}(K(\ZZ/2, n+2); \ZZ/24)$
		\item $\dots$
	\end{enumerate}
\end{ex}

If $X$ is a spectrum, we can also try to construct Postnikov towers.
However, especially if $X$ is not bounded below, there may not be a natural ``starting point'' for the Postnikov tower.
Instead, we fix a lower truncation $X\angles{r}$ and build a Postnikov tower for this out of EM spectra.
Considering these towers for all $r$, we end up with a 2-parameter family of $k$-invariants $k^{r,s}$.
These can be difficult to compute!
However, these produce \emph{collections} of cohomology operations.

\subsection{3/13 -- Addendum}

The secondary cohomology operations above can be understood in terms of Toda brackets.
Let's consider this for the Postnikov tower of $\SS$.
Given $x: X \to K(\ZZ, n)$ such that $\Sq^2 \circ x = 0$, the universal properties of cofibers produce a ``Toda bracket'' $\Sigma X \to K(\ZZ/2, n+4)$.
Via the loop-suspension adjunction, this gives a map $X \to K(\ZZ/2, n+3)$.
This is our secondary cohomology operation!

\section{3/13 (Gabriel Beiner) -- The Atiyah-Hirzebruch Spectral Sequence}

\subsection{Overview of the sequence}

\begin{thm}[Whitehead]
	Let $E$ be a spectrum and $X$ a finite CW complex.
	There are spectral sequences
	\begin{align*}
		E^2_{p,q} &= H_p(X; E_q(\pt)) \Rightarrow E^\infty_{p,q} = \gr(E_{p+q}(X)) \\
		E_2^{p,q} &= H^p(X; E^q(\pt)) \Rightarrow E_\infty^{p,q} = \gr(E^{p+q}(X)).
	\end{align*}
\end{thm}

These spectral sequences were originally discovered by Whitehead but was first published by Atiyah and Hirzebruch.
Thus the spectral sequences here are known as the \emph{Atiyah-Hirzebruch spectral sequences}.

\begin{rmk}
	\begin{enumerate}
		\item Given a fibration $F \to X \to B$, we obtain an \emph{Atiyah-Hirzebruch-Serre spectral sequence}
			\[
				H_*(B; E_*(F)) \Rightarrow E_*(X).
			\]
			The theorem above handles the case of $\pt \to X \to X$.
		\item We can generalize to the case where $X$ is a spectrum, but there are convergence issues if $X$ is unbounded.
		\item Spectral sequences arise from filtrations or more generally Massey exact couples.
			There are a few ways to use this to produce the AHSS.
			Cartan and Eilenberg obtained the AHSS using the filtration of $X$ by skeleta.
			Maunder obtained the AHSS instead using the Postnikov tower of $E$.
		\item Maunder's approach implies that the differentials in the AHSS are the Postnikov $k$-invariants of $E$.
		\item If $E$ is a ring spectrum, then the cohomological AHSS is multiplicative: each page has a multiplication and the differentials are derivations.
			This is fairly restrictive: for $H\ZZ/2 \to H\ZZ/2$, the derivations are Milnor primitives $Q_i = [\Sq^{2^i}, Q_{i-1}] \Sq^1$.
		\item The AHSS is natural / functorial in $X$ and $E$.
			In particular, isomorphisms on homology (with $\ZZ$ coefficients) induce isomorphisms on generalized homology.
		\item The $\QQ$-localization of any spectrum is a wedge sum of shifted $H\QQ$'s (if we ignore the ring structure).
			Thus $\QQ$-localizing the AHSS creates a spectral sequence for which all differentials are trivial.
			We end up with $E^*(X) \otimes \QQ \simeq H^*(X; \QQ) \otimes E^*(\pt)$.
			This recovers the Chern character isomorphism for $K$-theory: $\KU(X) \otimes \QQ \simeq H^*(X; \QQ) \otimes \KU^*(\pt)$!
	\end{enumerate}
\end{rmk}

\begin{ex}
	For $E = \KU$, the $E_2$ page of the cohomological AHSS has odd rows all zero and even rows given by $H^*(X)$.
	We can compute
	\[
		\KU^*(\CC\PP^n) = \begin{cases}
			\ZZ^{n+1} & * \textrm{ even} \\
			0 & * \textrm{ odd}
		\end{cases}
	\]
	because all differentials are forced to be trivial.
	The multiplicative structure can be found as $\KU^*(\CC\PP^n) = \ZZ[\gamma] / \gamma^{n+1}$ where $\gamma = [\Oc(-1)] - 1$.
	It is a good exercise to compute $\KU(\RR\PP^n)$ and $\KU(\RR\PP^3 \times \RR\PP^4)$.
	This latter computation can be performed using K\"unneth and implies that the first nontrivial differential of the AHSS is $d_3 = \beta \circ \Sq^2 \circ r: H^*(X; \ZZ) \to H^{*+3}(X; \ZZ)$.
\end{ex}

\begin{ex}
	For $\KO$, the $E_2$ page of the cohomological AHSS has rows given by $0$, $H^*(X; \ZZ/2)$, or $H^*(X; \ZZ)$ according to Bott periodicity.
	The differentials can be computed using Steenrod squares.
\end{ex}

\subsection{Steenrod's problem}

General spectral sequence theory gives an ``edge morphism'' $E^\infty_{*,0} \rightarrow E^2_{*,0}$.
If the spectral sequence degenerates, then the edge morphism is surjective.
For the AHSS, this is $E_*(X) \to H_*(X; \pi_0(E))$ induced by the Hurewicz map $E \to H\pi_0(E)$.

\begin{ex}
	A theorem of Thom says that $\MO$ is a wedge sum of shifted $H\ZZ/2$'s, i.e.\ there are no Postnikov invariants.
	Thus the AHSS degenerates and we get $\Omega^\Orm_*(X) \simeq H_*(X; \Omega^\Orm_*(\pt))$.
	The edge morphism is $\Omega^\Orm_n(X) \to H_n(X; \ZZ/2)$ given by $(f: M^n \to X) \mapsto f_*[M]$.
	It follows that all homology classes mod $2$ can be realized by unoriented bordism classes.
	This resolves Steenrod's problem for $\ZZ/2$ coefficients!
\end{ex}

\begin{ex}
	What happens for $\MSO$?
	Localizing $\MSO$ at $p = 2$ gives a wedge of $H\ZZ/2$'s and $H\ZZ_{(2)}$'s.
	Localizing $\MSO$ at $p \neq 2$ gives a wedge of Brown-Peterson spectra.
	One can compute the groups $\Omega_*^\SO$ and the corresponding differentials for small $*$.
	We get $\Omega_i^\SO(X) = H_i(X)$ for $0 \leq i \leq 3$ and $\Omega_4^\SO(X) = H_4(X) \oplus \ZZ$.
	These methods can be used to find a specific example where Steenrod's problem fails (it lives in $H^7(B\ZZ/3^2)$).
\end{ex}

\section{3/20 (Swapnil Garg) -- The Adams Spectral Sequence}

\subsection{Steenrod squares}

Let $\Ac = \Ac_2$ be the Steenrod algebra $[H\FF_2, H\FF_2]$.
This has an additive basis given by $\Sq^I = \Sq^{i_1} \Sq^{i_2} \dots$ where $I$ is \emph{admissible}: $i_k \geq 2 i_{k+1}$ for all $k$.\footnote{For Steenrod algebras $\Ac_p$ with $p \neq 2$, we also have to include Bockstein operations.}
The \emph{excess} $\operatorname{exc}(I)$ is $\sum_k (i_k - 2 i_{k+1})$.

\begin{thm}[Serre]
	The cohomology ring $H^*(K(\FF_2, n), \FF_2)$ is a polynomial algebra in $\Sq^I e_n$ for $\operatorname{exc}(I) < n$.
\end{thm}

\subsection{The Adams resolution}

To compute things, we can use fibrations together with the Leray-Serre spectral sequence.
One way to get these fibrations is by killing homotopy groups.
We use the ``Whitehead tower'' construction: if the first nontrivial homotopy group of $X$ is $\pi$ in degree $n$, we get a map $X \to K(\pi, n)$.
The homotopy fiber of this map is called $X|_{n+1}$.
We set up a spectral sequence to compute the homology groups of $X|_{n+1}$ and use these to compute $\pi_{n+1}(X|_{n+1}) = \pi_{n+1}(X)$.
In general, this is hard to use!

The idea of Adams is to kill $x \Sq^i(x)$ as well.
These give resolutions of spaces giving a free $\Ac$-resolution of $H^*(X)$.

\begin{dfn}
	An \emph{Adams resolution} of $X_0 = X$ is a sequence of spaces and maps
	\[
		\begin{tikzcd}
			\dots \rar["i_2"] & X_2 \rar["i_1"] & X_1 \rar["i_0"] & X_0
		\end{tikzcd}
	\]
	such that each $\cofib(i_s)$ is a sum of shifts of $H\FF_p$ and each $H^*(\cofib(i_s)) \to H^*(X_s)$ is an epimorphism.
\end{dfn}

The Adams resolution gives a free $\Ac$-resolution
\[
	\begin{tikzcd}
		H^*(X_0) & H^*(\cofib(i_0)) \lar & H^*(\Sigma \cofib(i_1)) \lar & H^*(\Sigma^2 \cofib(i_2)) \lar & \dots.
	\end{tikzcd}
\]

\subsection{The Adams spectral sequence}

We can use the Adams spectral sequence to ``compute'' $[X, Y]$:
\[
	E_2^{s,t} = \Ext_\Ac^{s,t}(H^* Y, H^* X) \Rightarrow [X, \hat{Y}_{(p)}]
\]
Here $\Ext^{s,t}$ refers to the $t$th component of $X^s$.
The differential here is different from Serre's convention: it is a map $E_r^{s,t} \to E_r^{s+r,t+r-1}$.
Here $\hat{-}_{(p)}$ is $p$-completion, which satisfies
\[
	\pi_i(\hat{Y}_p) = \widehat{\pi_i(Y)}_{(p)}
\]

Let $X = Y = \SS$.
Then the Adams spectral sequence is multiplicative and gives the composition product in stable homotopy groups of spheres.

\end{document}
